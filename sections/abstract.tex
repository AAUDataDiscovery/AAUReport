\pdfbookmark[0]{Abstract}{label:Abstract}%
  \addtolength{\hoffset}{0.5\evensidemargin-0.5\oddsidemargin} %set equal margins on the frontpage - remove this line if you want default margins
  \noindent%
  \begin{tabular}{@{}p{\textwidth}@{}}
    \begin{flushleft}
    \Huge{\textbf{
      Abstract
    }}
    \end{flushleft}
  \end{tabular}
Data discovery is a process of collecting and evaluate data from different sources in order to understand trends and patterns. The insights obtained during this proceess can be essensial in answering highly important business questions, improving the overall business operations, or just to prevent future treaths.
\vspace{2mm} %5mm vertical space
\\ One approach used by multiple business in the world's market is data exploration alongside visual analytics. Those are the primarly used methods for searching interesting correlations, trends, patterns, and anomalies that could require further investigation. Further on, the results can be used alongside visual analytics tools to represent the data on charts or diagrams. Overall, the main goal of Data Discovery is to represent the data in such a way that can be easily shared and understand by everyone.



%\section*{How is Data Discovered?}
%The process of data discovery can be quite debatable, because there isn't a step by step process on how to do it. But the most approached method is usually as follows:
%\begin{itemize}
%  \item \textbf{Step 1: Identify the needs.} Effective discovery begins with a clear purpose. One important mission on this step is to assess which kind of data should be used.
%  \item \textbf{Step 2: Combine multiple relevant sources.} In order for the data discovery process to be effective and efficient, is it important to incorporate as many sources into your operation. Analyzing vast amount of data will guarantee to deliver different outcomes as every piece of data tells a different story.
%  \item \textbf{Step 3: Prepare the data.} This step focus mostly in cleaning the data and preparing it so that can it be easily analyzed by organization members.
%  \item \textbf{Step 4: Analyze the data.} In the end, all the cleansed up data can now be used by business leaders to gain a complete view of their operations and solve the operational riddles that stand in the way of efficiency.
%\end{itemize}
%\pagebreak
%\section*{Research Objectives}
%In the present-day nearly all companies are collecting huge amount of data from their customers, from their market, distributors and so on. Data discovery provides the business leaders and their teams with a better overvierw of their operations so they can better understand and overcome any issues.
%\vspace{5mm} %5mm vertical space
%\\ Some of the biggest issues in the world's market working with data are:
%\begin{itemize}
%  \item \textbf{Storage.} Data needs to be stored. A big amount of data will require a substantial storage cappacity, which can increase the cost managment.
%  \item \textbf{Speed.} Additionally, data needs to be analyzed and cleaned during the process of data discovery, therefor a powerful system will be required that can manage to read that vast amount of data.
%  \item \textbf{Security.} Another issue in correlation with storage is security. Leaking data or information can be very costy for a company, therefor data encryption or allowing certaion users to work with the data should be another priority.
%  \item \textbf{Quality.} Having an huge amount of data won't guarantee to provide a good outcome of the discovery process. The quality of data should be anouther focus when defining the goals of the operation.
%  \item \textbf{Visual.} Diplaying data in a chart or diagram will help the people without any background knowledge of the whole data to better understand the topic. Displaying data in a visual way can also make spotting trends more efficiently.
%\end{itemize}
%\vspace{5mm} %5mm vertical space
%
%\clearpage