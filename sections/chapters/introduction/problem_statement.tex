\subsection{Problem}
Storing a lot of data can be very helpful as times goes on, but at the same time it also generates a lot of problems. First, and the most obvious problem is the size. Data requires storage space, big amount of data will require substantial memory space where it can be stored. Another issues generated by this practice is that, at some point, the data has to be analyzed in order to be useful. Analyzing the data will require a lot of time and can be very difficult depending on the amount of it. 
\vspace{5mm} %5mm vertical space
\\Not understanding the data can also be a typical problem. As an example, let's take the most used tool for data discovery which is "Microsoft Power BI". Using this tool, we can import our data file, and the tool will use the data provided to create charts or diagrams with our data. Going back to our problem, doing such a thing will be totally useless and a waste of time as we don't have a base idea on what to do with our data, and generating a couple of schemes won't help us discover anything.

\subsection{Background}
As each year pass, we can cleary see the growing and the demand of our digital industry. As the recent studies done by Eurostat, \textit{"In 2021, almost nine out of ten (89 \%) individuals in the EU, aged between 16 and 74 years, used the internet (at least once within the three months prior to the survey date)"}. 
\vspace{5mm} %5mm vertical space
\\Nowadays is more common to store and gather the data digitally and many institutions are already gathering data in an automatic way from different sources (ex: surveys after buying a product from their web store). 
\vspace{5mm} %5mm vertical space
\\The only thing that remains is to analyze this data.

\subsection{Relevance}
But why is data discovery important? 
\vspace{5mm} %5mm vertical space
\\Analyzing the data gathered from your operations could be the helping hand in defeating your competitors. Let's take as an example the data gathered from your reviews provided by your customers in the past two months. Assessing this feedback can help you in spoting your weakness but also your strengths. As another example, you can also analyze the most sold product for each month. Doing this operation can help you in spotting patterns or trends that can be used in future business operations. 
\vspace{5mm} %5mm vertical space
\\With that in mind, data discovery can give you the upper hand in your field. Its interpretation can be easily used as an advantage, whether in the form of better customer experiences or increasing the overall profit.

\subsection{Objectives}
In this project, our objective is to develop and provide the user with a fast and reliable tool that will provide useful intelligence. As we are working with unstructured data. The outcome of our tool can later be used on creating graphical schemes on the data. Our main plan of the tool is to also allow the user to work with multiple data types. such as json file, xlsx, etc.
\vspace{5mm} %5mm vertical space
\\Overall, we believe that the definition of our problem statement would be something like: \textit{We need to find a way to design a tool that it's able to extract insightful data from unstructured data.}

\subsection{Research questions}
Throughout the whole research process, we bumped into the following questions, that we aiming to provide an answer 
\begin{itemize}
    \item What's the best method to "tag" or to group a collection of related data?
    \item How can we design our tool to find similarities from two related collections of data?
    \item What would be the best approach to display the insightful information gathered fron the data to the user?
    \item Is there any tool currently on the market with the same main objective?

\subsection{Context}
Companies often have to deal with large amount of unstructured data.
This data can come from many different sources (such as sensors, systems running 24/7, and users) in a variety of
different formats.
As technology evolves, the rate at which data is gathered only ever increases, as computer systems gain new ways to
extract information from the environment, and people leave a digital footprint in almost every daily activity.

\subsection{Issue}
The problem with this development is that as a data set grows, it often loses its meaning.
Data is seldom stored in a perfect manner, with annotations explaining what each entry represents and what can we
expect from a file with millions of rows.
The sizes of these files often make it impossible to manually analyze them and determine the nature of the data stored within.
When data analysts take these files and input them into a system like Microsoft Power BI, they produce graphs that often fail to
provide any real insights. 
It is pointless to look at numbers in a diagram if we don't know what these numbers mean.

\subsection{Objectives}
The purpose of this project is to develop a tool that can extract valuable insights out of a large amount of unstructured data.
Essentially, the final product should be used by a customer before inputting the data into a data visualization tool like Power BI.
The additional information (metadata) produced by our system should be enough for an analyst to determine the kind of data
he is dealing with: whether it was produced by sensors or humans, the period time it was extracted in, and so on.
We will be using similarity techniques, a formal representation of metadata and labelling approaches to determine, as
accurately as possible, what kind of data we are presented with in each scenario.

\bigbreak

With this in mind, we define our project's problem statement as follows:
\textit{How can we design and implement a tool capable of analysing large amounts of data and extract relevant
information about its content?}

To be able to answer this problem statement, we consider these auxiliary questions:
\begin{itemize}
    \item What are the existing tools in the industry capable of?
    \item How can metadata help us produce meaningful information from seemingly arbitrary data?
    \item How can we categorize (or tag) datasets?
    \item How can we compare datasets to identify similarities?
    \item How can we combine and visualize relevant results about a collection of datasets?
\end{itemize}