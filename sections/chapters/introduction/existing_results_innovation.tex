\subsection{Existing Solutions on the Market}\label{subsec:existing_solutions_on_the_market}
Initial market research led us to discover many already developed solutions for the implementation of a data catalog.
We will look at three examples of data catalog solutions and outline their key features.
Afterwards, we will present the main selling points of our own solution, explaining how it innovates the already
established data discovery market.

\subsubsection{Alation Data Catalog~\cite{AlationDataCatalog}}
Alation's Data Catalog solution is focused on collaboration and connectivity.
It has five broad, highlighted features:
\begin{itemize}
    \item A wide variety of data sources, such as relational databases, cloud data lakes, and file systems.
    \item ``Seamless'' collaboration and realtime communication.
    \item Natural language search.
    \item The possibility to set rules and enforce policies.
    \item APIs to connect data sources with off-the-shelf BI tools.
\end{itemize}

\subsubsection{IBM Watson Knowledge Catalog~\cite{IBMWatsonKnowledgeCatalog}}
IBM's catalog highlights the benefits of AI-powered data discovery.
Its main feature is its highly extensive search function.
It allows users to search in plain English for tags or contents of their datasets.
It also keeps a history of searches and offers recommendations based on it.
Like the Alation Data Catalog, IBM also permits the creation and enforcement of rules and policies.
Finally, IBM's Data Catalog creates graphs to visualize the data, but that function goes beyond the scope of our project
(recall that our system is meant to provide meaning out of data such that it can be easily manipulated later, using data
visualization tools).

\subsubsection{Google Data Catalog~\cite{GoogleDataCatalog}}
The Google Data Catalog is a service contained within the Dataplex product~\cite{GoogleDataplex}.
Its functionality is rather simple, as it offers the possiblity to catalog data, generate metadata, and manually tag it.
The primary selling point of Google Data Catalog is that it automatically extracts and catalogs data from its other
products, such as BigQuery, Pub/Sub, and Cloud Storage thus leveraging its incredibly popular and vast library of
data solutions.

\bigbreak

This information can help us determine what our project can innovate on the data cataloging scene.
First of all, we need to include some essential features for our solution to be relevant.
Among these, our system should be capable of processing various data formats, as real world sources generally produce
(examples: CSV, JSON, XLS).
Additionally, metadata should be at the forefront of our system's operations.
Going a bit deeper, it appears that none of the existing products presented above focus on building connections between
pieces of data.
Automatic suggestions for how different datasets relate to each other, how similar they are, and whether they might
belong in the same category, or might have the possibility of being merged together, is an incredibly useful feature
for a data discovery system to offer.
For example, a company that deals with sensors, receives dirty data from various sources, and could find it difficult
to put it all together.
A tool with these capabilities could inform the user if two sets of sensor readings are closely related, as they might
be duplicate, or extracted at the same time, etc.
This can extend to tagging.
It is useful to allow the user to set manual tags, but recommending appropriate tags for newly cataloged datasets has
the potential to speed up and simplify the organizational process.
Naturally, tag suggestions would have to rely on already classified datasets (therefore requiring an already established
collection of data), but this feature is still an example of a system becoming more efficient and easy to use over time.
These features will become the highlights of our product and, put together and extended with additional functionality,
can truly bring a fresh option on the market of data catalogs.