\section{Metadata}
\subsection{The Necessity for Formalized Persistence}

Having the results available from the Data Discovery tool only during runtime is severely limiting.
Therefore, the output of the library must be persisted for later analysis.
Moreover, a standardized
metadata format is useful for data representation during runtime.
\newline

A practical solution must adhere to several criteria, namely:
\begin{enumerate}
    \item It must use a pre-existing format.
    \item Low overhead.
    \item Minimal number of dependencies.
    \item Human readable.
\end{enumerate}

The requirement for the pre-existing format is an obvious one.
By using standardized formats such as XML we eliminate the need for custom tools to read the
data generated.

The second point is regarding performance considerations, generating and persisting the metadata must be
with as little disruption as possible, especially considering that IO resources may be limited as the
analysis is running.

The third condition is again concerned with portability.
Ideally, the users of the library shouldn't be
forced to install any specialized applications just to read the metadata.
Enforcing this point also reduces the complexity of the data discovery tool


The last requirement, while not the most significant, enables the users to quickly understand the
state of the system, without having to rely on any helper tools.
