\chapter{Conclusion}\label{ch:conclusion}

\section{Answering The Problem Statement}

\begin{itemize}
    \item \texit{How can we design and implement a tool capable of analysing large amounts of data and extract relevant
    information about its content?} \newline
    Build metadata using a consistent schema and manage it in a scalable manner via a file monitoring system.
    Compile a powerful library of data exploration and matching algorithms.
    \item \texit{What are the existing tools in the industry capable of?} \newline
    We determined that current data catalogs are capable of storing files, organizing them, but the tagging has to be done manually.
    Also, they don't suggest logical connections and similarities between data, and they don't always support any format of
    source material.
    \item \texit{How can metadata help us produce meaningful information from seemingly arbitrary data?} \newline
    By exposing properties that are not explicitly stated in the table rows and comparing them using the data matching library.
    \item \texit{How can we categorize (or tag) datasets?} \newline
    Allow the user to set meaningful tags, then use matching algorithms to automatically suggest appropriate tags for new
    datasets.
    This approach becomes more efficient as the catalog grows in size.
    \item \texit{How can we compare datasets to identify similarities?} \newline
    With various analysis algorithms, some leveraging the dataframe contents at the cost of performance, while others making
    use of information available in the metadata, saving execution time.
    \item \texit{WHow can we combine and visualize relevant results about a collection of datasets?} \newline
    By designing a consistent metadata schema and displaying it, both in static documents, and in an interactive web client.
\end{itemize}


\section{Reflection}
We started this project with the intent of building a tool that can take a dataset as input
and produce some sort of insight into the data as an output.

Initially, we only had a faint idea of what it means to \("\)understand\("\)
the data.
We were unsure if it was even realistic to create such a tool within the project's constraints.

Therefore, we started small.
Communicating basic things, such as column names, datatypes etc.
were easy, but they were arguably among the most useful information we could relay to the users.

At this point, we had a foothold, and we had a concept of what was next.
We did slightly less obvious things, such as calculating the distance between strings or
calculating correlations between columns.

This simple toolset we had on our hands was already able to find implicit connections between
otherwise disconnected files.
In essence, we were closing in on our goal of providing insight into an otherwise, seemingly chaotic
dataset.

While finding meaning in the subject data was the core part of our project,
it was by no means the only part of it.

We were investigating ways to communicate our findings.
The metadata generated as a result of these analyses will provide the users with useful information
they can use to make better decisions.
Moreover, it's done in such a way that the metadata may be passed to third-party tools and
its functionality can be extended beyond its original purpose.

One example of this extensibility is the visualizer, a simple tool we created, that only
uses the metadata to create intuitive graphics of the structure of the files, and
every piece of insight the library has outputted.

But we also looked into other datasets,
because how could we create a tool, that analyzes arbitrary datasets, if
we didn't have a solid understanding of what was actually out there.

This investigation later led to the idea of tagging.
If we can reason that datasets will often fall into standard categories,
then we can extend our tool to recognize these categories without human intervention.

Of course, the initial implementation of the tagging is simple, reliant again on
string distances and other rudimentary algorithms.
But even with such a straightforward implementation, it performed unexpectedly well.
Possibly providing the most intuitive form of understanding for humans.

Therefore, in conclusion, not only did we create a tool that provides insight into arbitrary data,
we organically created a tool that makes the whole analysis process easier.


\section{Future Considerations}
Of course, as with any other project, this one had its limitations as well, both
in terms of time and personnel.
We'd have loved to implement more data-matching algorithms,
work on clustering, and other quality-of-life tools, such as an extended visualizer.

The biggest opportunity for improvement would be to make the library more extensible,
and configurable.

Its users should be able to write their own matching or tagging algorithms and have
them used by the core library, preferably without the users having to have an in-depth
understanding of the library's inner workings.

Another opportunity that couldn't be implemented on time was the live analysis.
While the file daemon was available on time, at this point running the matching
and tagging algorithms is a static process, executed manually by the users
whenever they need it.


\section{Discussion}

When we received the project's description it was intentionally vague.
This was not a problem, it was an opportunity.
We were expected to make the project our own.
An unexpected side benefit of this vagueness was that not only, we as a team
made the project our own, but each team member made the project their own, in their way.

Because we are all different people, with different interests, when we looked at the problem, we saw different parts of a picture.
Therefore, we diverged.
Some of us stayed on topic and focused on data matching.
Some focused on performance, while others focused on infrastructure, and scaffolding or
just acting as floating agents, helping out wherever possible.

Ultimately, however, when all the pieces fell together, we ended up with something coherent, the whole picture.
Since our interests and focuses were so varied, we also ended up learning from each other,
if for nothing else, just to keep up with each other.

To summarize, from an indistinct kernel of an idea, we ended up with something
that could objectively be considered useful.
While in the meantime learning new concepts and ideas we wouldn't normally encounter in our
everyday life.

